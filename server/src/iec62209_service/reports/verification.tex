\documentclass{article}
\usepackage[T1]{fontenc}
\usepackage[utf8]{inputenc}
\usepackage{lmodern}
\usepackage{textcomp}
\usepackage{lastpage}
\usepackage{geometry}
\geometry{tmargin=2cm,lmargin=2cm,includeheadfoot=True}
\usepackage{float}
\usepackage{hyperref}
\usepackage{ragged2e}
\usepackage{graphicx}
\usepackage{subcaption}
\usepackage{longtable}
\usepackage[yyyymmdd,hhmmss]{datetime}
\usepackage{enumitem}
\usepackage{placeins}
\usepackage[parfill]{parskip}
\usepackage[mainaux]{rerunfilecheck}

\begin{document}
\flushright
\begin{Large}
\textbf{IEC 62209-3 Validation Results} \\
\vspace{3pt}
\textbf{Critical Data Space Search}
\end{Large} \\
\begin{small}
\today \\ \currenttime
\end{small}
\flushleft

\section{Executive Summary}\label{sec:exec_summary}
The SAR measurement system validation procedure described in IEC 62209-3~\cite{standard} is a three step procedure the consists of a) GPI model creation, b) model confirmation, and c) the critical data space search. This automatically generated document reports on the outcome of the \textit{critical data space search}.

\textbf{\textit{The SAR measurement system described in Table~\ref{tab:system} passes the critical data space search. In combination with the model confirmation step, the system can therefore be considered successfully validated.}}

The results of this procedure are shown in Table~\ref{tab:summary}.

\input{metadata}

\input{summary}

\section{Introduction}\label{sec:start}
\subsection{GPI Model}
The GPI model is a model that describes the expected measurement error and uncertainty for the given SAR measurement system of interest as a function of exposure parameters.

\subsection{Purpose of Report}\label{sec:intro}
This report provides the results of the \textit{critical data space search} step for the SAR measurement system validation procedure described in IEC 62209-3~\cite{standard}.

The goal of this step is to test the SAR measurement system performance in regions of the exposure configuration space where the system is likely (5~\% or more) to exceed the acceptable measurement error according to the GPI model. The selection of these critical test conditions is driven by both, the need to explore the relevant exposure condition space, as well as the desire for increased scrutiny of regions where the system is likely to fail to meet the expected measurement performance.

This report has been automatically generated by an online-accessible, GUI-based validation application (\url{http://sarvalidation.site/} version 1.0.5) using measured data obtained with the SAR measurement system described in Table~\ref{tab:system}.

Background and additional information on the methodology can be found in the open-access paper~\cite{gpi-paper}. The open-source software leveraged by the online application is provided with IEC 62209-3 and can be found at
\url{https://github.com/ITISFoundation/publication-IEC62209}.

\subsection{Critical Data Space Search}\label{sec:procedure}
SAR measurement system validation in IEC 62209-3~\cite{standard} consists of these three steps:

\begin{enumerate}[label=\alph*)]
\item Gaussian process interpolation (GPI) model creation
\item GPI model confirmation
\item Critical data space search
\end{enumerate}

The results of the \textit{critical data space search} (c) are reported in this document. The process involves the following procedure (see Annex D, Clause D.4.4):

Critical data space search -- The test lab performs a set of measurements of critical test conditions identified based on the confirmed GPI model. The purpose is to identify potential regions of the exposure configuration space in which the measurement system has a high likelihood of exceeding the accepted error margin. The critical test configuration selection algorithm uses the GPI model to estimate the failure likelihood. It  maximizes test space coverage and and prioritizes conditions with high failure probability. The system validation is considered successfully concluded if no critical cases are identified or if all measurements of the critical cases agree with the numerical target values within the acceptance threshold. If any measurement result exceeds that threshold, the system cannot be considered validated and is not acceptable for use.

\subsection{Success Criterion}\label{sec:quantities}

The following criterion must be met for the system to successfully pass the critial data space search:

\begin{itemize}
\item \textbf{acceptance}\\ for the system to be considered acceptable for use in accordance with IEC 62209-3, all test configuration SAR deviations ($\Delta SAR_j, j$ = 1 to $N$) between the measured SAR values and the numerical SAR target values (specified by IEC) must be within the acceptance criteria given in Clause D.4.7 of~\cite{standard}:

\[
-U < r_{s,j} < +O, ~~~~~~~~ j = 1~..~N
\]

where $r_{s,j}$ is the linear deviation between the measured SAR ($SAR_{m,j}$) and the numerical target ($SAR_{num,j}$) given by:

\[
r_{s,j} = 100~\% \times \Big(\frac{SAR_{m,j} - SAR_{num,j}}{SAR_{num,j}}\Big)
\]

and the deviation in dB is $\Delta SAR_j = 10 \times log_{10} (r_{s,j})$. The error bounds [-$U$, +$O$] are given by:

\[
+O = 2 \times u_s + 15~\%
\]

\[
-U = -100 \times \frac{2 \times u_s + 15~\%}{100 + 2 \times u_s + 15~\%}
\]

where 2 x $u_s$ is the reported measurement uncertainty of the SAR measurement with a 95~\% confidence level.

When expressed in dB, the error bounds are equal, meaning that $10 \times log_{10}(O) = 10 \times log_{10}(U)$, so the requirement simplifies to:

\[
\big|\Delta SAR_j\big| = 10 \times log_{10}(+O)
\]
\end{itemize}

\section{Critical Cases}

\subsection{Input Parameters}
The set of exposure conditions for the critical testing of the measurement system described in Table~\ref{tab:system} was generated and measured on the system using the input parameters of Table~\ref{tab:params}.

\subsection{Limits of Relevant Exposure Parameter Space}
The test configurations for the critical testing of the measurement system detailed in Table~\ref{tab:system} were chosen within the parameter range defined in Table~\ref{tab:params}, which therefore defines the extent of the exposure parameter space for which the GPI model can be considered to have been critically examined.

\input{sample_parameters}

\FloatBarrier
\subsection{Performance on Acceptance Criteria}
The test data for the critical GPI model exploration are shown in Figure~\ref{fig:critical-acc}. Thecomplete details on the exposure conditions and measurement results are tabulated in Table~\ref{tab:test} in Appendix~\ref{sec:critical-data}.
The obtained deviations ($\Delta SAR_{10g}$) of the critical test configuration measurements from the target values are shown in Figure~\ref{fig:critical-acc}, along with the acceptance thresholds. The pass/fail result is shown in Table~\ref{tab:acceptance}.

\input{acceptance}

\begin{figure}[H] \centering
\includegraphics[width=\textwidth]{images/critical-acceptance.png}
\caption{Deviations of the measured $SAR_{10g}$ from the target values for the critical test configurations. The deviations are compared to the acceptance thresholds.} \label{fig:critical-acc}
\end{figure}

\FloatBarrier

\newpage

\begin{thebibliography}{9}
\bibitem{standard}
IEC 62209-3, ``Measurement procedure for the assessment of specific absorption rate of human exposure to radio frequency fields from hand-held and body-mounted wireless communication devices - Part 3: Vector measurement-based systems (Frequency range of 600 MHz to 6 GHz)'', Committee Draft, February 2023.
\bibitem{gpi-paper}
C. Bujard, E. Neufeld, M. Douglas, J. Wiart, N. Kuster, ``A Gaussian-process-model-based approach for robust, independent, and implementation-agnostic validation of complex multi-variable measurement systems: application to SAR measurement systems,'' online \url{https://arxiv.org/abs/2211.12907}, uploaded April 12, 2023.
\end{thebibliography}

\FloatBarrier
\newpage
\appendix
\section{Critical Configurations and Measurement Outcomes} \label{sec:critical-data}

\input{sample_table}

\end{document}
