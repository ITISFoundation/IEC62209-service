\documentclass{article}
\usepackage[T1]{fontenc}
\usepackage[utf8]{inputenc}
\usepackage{lmodern}
\usepackage{textcomp}
\usepackage{lastpage}
\usepackage{geometry}
\geometry{tmargin=2cm,lmargin=2cm,includeheadfoot=True}
\usepackage{float}
\usepackage{hyperref}
\usepackage{ragged2e}
\usepackage{graphicx}
\usepackage{subcaption}
\usepackage{longtable}
\usepackage[yyyymmdd,hhmmss]{datetime}
\usepackage{enumitem}
\usepackage{placeins}
\usepackage[parfill]{parskip}
\usepackage[mainaux]{rerunfilecheck}

\begin{document}
\flushright
\begin{Large}
\textbf{IEC 62209-3 Validation Results} \\
\textbf{GPI Model Creation}
\end{Large} \\
\begin{small}
\today \\ \currenttime
\end{small}
\flushleft

\section{Executive Summary}\label{sec:exec_summary}

The SAR measurement system validation procedure described in IEC 62209-3~\cite{standard} is a three step procedure that consists of a) GPI model creation, b) model confirmation, and b) the critical data space search. This automatically generated document reports on the outcome of the \textit{GPI model creation}.

\textbf{\textit{The SAR measurement system described in Table~\ref{tab:system} successfully completed the GPI model creation step.}}

The results of the two test criteria of the GPI model creation are shown in Table~\ref{tab:summary}.

\input{metadata}

\input{summary}

\section{Introduction}\label{sec:start}

\subsection{Purpose of Report}\label{sec:intro}
This report provides the results of the\textit{ GPI model creation} step for the SAR measurement system validation procedure described in IEC 62209-3~\cite{standard}. The GPI model is a model that describes the expected measurement error and uncertainty for the given SAR measurement system of interest as a function of exposure parameters. This report has been automatically generated by an online-accessible, GUI-based validation application (\url{http://sarvalidation.site/} version 1.0.6) using measured data obtained with the SAR measurement system described in Table~\ref{tab:system}.

Background and additional information on the methodology can be found in the open-access paper~\cite{gpi-paper}. The open-source software leveraged by the online application is provided with IEC 62209-3 and can be found at
\url{https://github.com/ITISFoundation/publication-IEC62209}.


\subsection{GPI Model Creation}\label{sec:procedure}
SAR measurement system validation in IEC 62209-3~\cite{standard} consists of these three steps:

\begin{enumerate}[label=\alph*)]
\item Gaussian process interpolation (GPI) model creation
\item GPI model confirmation
\item Critical data space search
\end{enumerate}

The results of \textit{GPI model creation} (a) are reported in this document. The process involves the following steps (see Clause D.4.4 of~\cite{standard}):

\begin{enumerate}[label=\arabic*)]
\item Test set generation -- The GPI software generates the measurement configurations (i.e. antenna types, frequencies, power levels, modulations, antenna locations, and antenna angles and distances) to be measured. To reduce the number of configurations (high dimensional parameter space with  millions of available exposure conditions) required to generate the GPI model, a smart algorithm selects configurations that sparsely, pseudo-randomly, but uniformly cover all parameter space dimensions.
\item Measurement -- The model generating party (system manufacturer, independent laboratory, or test laboratory) measures the test configurations on a reference SAR measurement system. The resulting model is valid only for the measured hardware and software version. Should outliers emerge, they are identified and the measurements double checked. The system is acceptable for use if all results are within the acceptance criteria specified in the standard.
\item Analysis and model creation -- The software analyses the measurement data and uses semivariograms and range-isotropisation to produce a global semivariogram and a GPI model. That model is then provided to the test laboratory for the independent model confirmation step, along with a report detailing the generation of the GPI model (this document).
\end{enumerate}

\subsection{Success Criteria}\label{sec:quantities}

Two criteria must be met for the model creation to be considered successful:

\begin{itemize}
\item \textbf{acceptance}\\ for the system to be considered acceptable for use in accordance with IEC 62209-3, all test configuration SAR deviations ($\Delta SAR_j, j$ = 1 to $N$) between the measured SAR values and the numerical SAR target values (specified by IEC) must be within the acceptance criteria given in Clause D.4.7 of~\cite{standard}:

\[
-U < r_{s,j} < +O, ~~~~~~~~ j = 1~..~N
\]

where $r_{s,j}$ is the linear deviation between the measured SAR ($SAR_{m,j}$) and the numerical target ($SAR_{num,j}$) given by:

\[
r_{s,j} = 100~\% \times \Big(\frac{SAR_{m,j} - SAR_{num,j}}{SAR_{num,j}}\Big)
\]

and the deviation in dB is $\Delta SAR_j = 10 \times log_{10} (r_{s,j})$. The error bounds [-$U$, +$O$] are given by:

\[
+O = 2 \times u_s + 15~\%
\]

\[
-U = -100 \times \frac{2 \times u_s + 15~\%}{100 + 2 \times u_s + 15~\%}
\]

where 2 x $u_s$ is the reported measurement uncertainty of the SAR measurement with a 95~\% confidence level.

When expressed in dB, the error bounds are equal, meaning that $10 \times log_{10}(O) = 10 \times log_{10}(U)$, so the requirement simplifies to:

\[
\big|\Delta SAR_j\big| = 10 \times log_{10}(+O)
\]

\item \textbf{nrmse $\le$ 25~\%}\\ the normalized root-mean-square error (nrmse) of the model must be less than or equal to 25~\%. If this criterion is not met, it is recommended not to continue with the model confirmation or critical data space search, as they will not be based on a good model.
\end{itemize}

\section{Model Creation}

\subsection{Limits of Relevant Exposure Parameter Space}
The test configurations were generated for measurement on the system detailed in Table~\ref{tab:system} according to the parameters in Table~\ref{tab:params}, which therefore define the extent of the exposure parameter space for which the GPI model can be considered to be relevant.

\input{sample_parameters}

\FloatBarrier
\subsection{Test Configurations}

Figure~\ref{fig:training-dist} illustrates how the test configuration sample used for GPI model construction is distributed along the different exposure parameter space dimensions. The complete details on the exposure conditions and measurement results are shown in Annex~\ref{sec:training-data}.

\begin{figure} \centering
\includegraphics[width=\textwidth]{images/model-creation-distribution.png}
\caption{Distribution of the test configurations showing how they uniformly, but pseudo-randomly, cover the exposure parameter space dimensions.} \label{fig:training-dist}
\end{figure}

\FloatBarrier
\subsection{Performance on Acceptance Criteria}
The obtained deviations ($\Delta SAR_{10g}$) of the test configuration measurements from the target values are shown in Figure~\ref{fig:creation-acc}, along with the acceptance thresholds. The raw data are tabulated in Table~\ref{tab:training} in the Appendix.
The pass/fail result is shown in Table~\ref{tab:acceptance}.

\input{acceptance}

\begin{figure}[H] \centering
\includegraphics[width=\textwidth]{images/model-creation-acceptance.png}
\caption{Deviations in $SAR_{10g}$ compared to the target values for the test configurations. The deviations are compared to the acceptance thresholds.} \label{fig:creation-acc}
\end{figure}

\FloatBarrier
\subsection{Model Fitting Quality}
The obtained SAR-error semivariogram is shown in Figure~\ref{fig:creation-variogram}. The upper graph shows the fit of the model to the model errors. The lower bar graph shows the distribution of the errors. The quality of that fit (quantified as nrmse) is relevant with regard to the GPI model quality. Figure~\ref{fig:creation-marginals} shows how the errors are distributed along the different dimensions of the parameter space (frequency, location, etc.). The pass/fail result is shown in Table~\ref{tab:nrmse}.

\input{gfres}


\begin{figure} \centering
\includegraphics[width=\textwidth]{images/model-creation-semivariogram.png}
\caption{GPI semi-variogram construction: fitting (top), and histogram of the lags available for the semi-variogram construction (bottom).} \label{fig:creation-variogram}
\end{figure}

\begin{figure} \centering
\includegraphics[width=\textwidth]{images/model-creation-marginals.png}
\caption{Marginals showing the distribution of measurement errors in the training data.} \label{fig:creation-marginals}
\end{figure}

\FloatBarrier
\begin{thebibliography}{9}
\bibitem{standard}
IEC 62209-3, ``Measurement procedure for the assessment of specific absorption rate of human exposure to radio frequency fields from hand-held and body-mounted wireless communication devices - Part 3: Vector measurement-based systems (Frequency range of 600 MHz to 6 GHz)'', Committee Draft, February 2023.
\bibitem{gpi-paper}
C. Bujard, E. Neufeld, M. Douglas, J. Wiart, N. Kuster, ``A Gaussian-process-model-based approach for robust, independent, and implementation-agnostic validation of complex multi-variable measurement systems: application to SAR measurement systems,'' online \url{https://arxiv.org/abs/2211.12907}, uploaded April 12, 2023.
\end{thebibliography}

\FloatBarrier
\appendix
\section{Training Data Set} \label{sec:training-data}

\input{sample_table}

\end{document}
