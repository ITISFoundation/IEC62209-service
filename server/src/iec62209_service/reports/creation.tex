\documentclass{article}
\usepackage[mainaux]{rerunfilecheck}
\usepackage[T1]{fontenc}
\usepackage[utf8]{inputenc}
\usepackage{lmodern}
\usepackage{textcomp}
\usepackage{lastpage}
\usepackage{geometry}
\geometry{tmargin=2cm,lmargin=2cm,includeheadfoot=True}
\usepackage{float}
\usepackage{hyperref}
\usepackage{ragged2e}
\usepackage{graphicx}
\usepackage{subcaption}
\usepackage{longtable}
\usepackage[yyyymmdd,hhmmss]{datetime}
\usepackage{enumitem}
\usepackage{placeins}

\begin{document}
\flushright
\begin{Large}
\textbf{IEC 62209-3 Validation Results} \\
\textbf{GPI Model Creation}
\end{Large} \\
System Operator: A+ Test Laboratory \\
\begin{small}
\today \\ \currenttime
\end{small}
\flushleft

\section{Executive Summary}\label{sec:exec_summary}

\textbf{\textit{The SAR measurement system described in Table~\ref{tab:system} passes the GPI model creation step. The two test results of the GPI model creation are shown in Table~\ref{tab:summary}.}}

\input{metadata}

\input{creation_summary}

\section{Introduction}\label{sec:start}

\subsection{Purpose of Report}\label{sec:intro}
This report provides the results of the GPI model creation step for the SAR measurement system validation that is described in IEC 62209-3~\cite{standard}. It has been automatically generated by validation software using measured data provided on the SAR measurement system described in Table~\ref{tab:system}. The measured data has been supplied by the measurement system manufacturer, the user of the measurement system, a third party, or a combination of these. The validation software can be accessed either using an online graphical user interface (available at \url{http://sarvalidation.site/}) or by running the open-source software provided with IEC 62209-3. Details of the methodology can be found in an open-access paper~\cite{gpi-paper}.


\subsection{GPI Model Creation}\label{sec:procedure}
SAR measurement system validation in IEC 62209-3~\cite{standard} consists of these three steps:

\begin{enumerate}[label=\alph*)]
\item Gaussian process interpolation (GPI) model creation
\item GPI model confirmation
\item Critical data space search
\end{enumerate}

The results of GPI model creation in step a) are reported in this document, which has the following steps:

\begin{enumerate}[label=\arabic*)]
\item Test set generation -- The GPI software generates the measurement configurations (i.e. antenna types, frequencies, power levels, modulations, antenna locations, and antenna angles and distances). To reduce the number of configurations (several million in total possible), an intelligent selection is made; e.g. 400 tests, depending on the measurement system parameters.
\item Measurement -- The system manufacturer (or independent laboratory or the test laboratory) measures the test sample on a reference SAR measurement system. The measurement is valid only for the hardware version or software version measured. Potential outliers are extracted and double checked. The system is acceptable for use if all results are within the acceptance criteria.
\item Analysis and model creation -- The software analyses the measurement data and uses semivariograms and range-isotropisation to produce a global semivariogram and a GPI model. The system manufacturer provides to the test laboratory the GPI model including a report detailing the generation of the GPI model.
\end{enumerate}

\subsection{Quantities of Interest}\label{sec:quantities}

Both of the criteria below must be met for the model creation to be successful.

\begin{itemize}
\item \textbf{acceptance}\\ for the training data, all deviations, $\Delta SAR$, between the measured SAR values and the numerical SAR target values (in IEC)  must be within the acceptance criteria for the system to be considered acceptable for use with IEC 62209-3.

\item \textbf{nrmse $\le$ 25~\%}\\ the normalized rms error (nrmse) of the model must be less than or equal to 25~\% for the model to be usable in the GPI model confirmation step. If this criterion is not met, it is recommended not to continue with the model confirmation or critical data space search, as they will not be based on a good model.
\end{itemize}

\section{Model Creation}

\input{sample_parameters}

\FloatBarrier
\subsection{Training Data}

Figure~\ref{fig:training-dist} shows how the training data set is distributed among the dimensions of the model. The full set of data with measurement results is shown in Annex~\ref{sec:training-data}.

\begin{figure} \centering
\includegraphics[width=\textwidth]{images/model-creation-distribution.png}
\caption{Distribution of the training data, showing how the points are distributed among the main dimensions.} \label{fig:training-dist}
\end{figure}

\FloatBarrier
\subsection{Acceptance Criteria}
The training data for the GPI model are shown in Figure~\ref{fig:creation-acc}. The raw data are tabulated in Table~\ref{tab:training} in the Appendix.
In Figure~\ref{fig:creation-acc}, each measured $SAR_{10g}$ is compared to the target value, and the deviation $\Delta SAR_{10g}$ is shown. The pass/fail result is shown in Table~\ref{tab:acceptance}.

\input{creation_acceptance}

\begin{figure}[H] \centering
\includegraphics[width=\textwidth]{images/model-creation-acceptance.png}
\caption{Deviations in $SAR_{10g}$ compared to the target values for the training data. The deviations are compared to the acceptance criteria.} \label{fig:creation-acc}
\end{figure}

\FloatBarrier
\subsection{Model Fitting}
The semivariogram of the model errors is shown in Figure~\ref{fig:creation-variogram}. The upper graph shows the fit of the model to the model errors. The lower bar graph shows the distribution of the errors. Figure~\ref{fig:creation-errors} shows the statistical distribution of the measurement errors in the training data, and Figure~\ref{fig:creation-marginals} shows how the errors are distributed along the different dimensions of the parameter space (frequency, location, etc.). The pass/fail result is shown in Table~\ref{tab:nrmse}.

\input{creation_fitting}

\begin{figure} \centering
\includegraphics[width=\textwidth]{images/model-creation-semivariogram.png}
\caption{Semi-variogram of the model.} \label{fig:creation-variogram}
\end{figure}

\begin{figure} \centering
\includegraphics[width=\textwidth]{images/model-creation-error-distribution.png}
\caption{Statistical distribution of the measurement errors in the training data.} \label{fig:creation-errors}
\end{figure}

\begin{figure} \centering
\includegraphics[width=\textwidth]{images/model-creation-marginals.png}
\caption{Marginals showing the distribution of measurement errors in the training data.} \label{fig:creation-marginals}
\end{figure}

\FloatBarrier
\begin{thebibliography}{9}
\bibitem{standard}
IEC 62209-3, ``Measurement procedure for the assessment of specific absorption rate of human exposure to radio frequency fields from hand-held and body-mounted wireless communication devices - Part 3: Vector measurement-based systems (Frequency range of 600 MHz to 6 GHz)'', Committee Draft, February 2023.
\bibitem{gpi-paper}
C. Bujard, E. Neufeld, M. Douglas, J. Wiart, N. Kuster, ``A Gaussian-process-model-based approach for robust, independent, and implementation-agnostic validation of complex multi-variable measurement systems: application to SAR measurement systems,'' online \url{https://arxiv.org/abs/2211.12907}, uploaded April 12, 2023.
\end{thebibliography}

\FloatBarrier
\appendix
\section{Training Data Set} \label{sec:training-data}

\input{training_sample_table}

\end{document}
