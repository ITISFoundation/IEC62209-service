\documentclass{article}
\usepackage[T1]{fontenc}
\usepackage[utf8]{inputenc}
\usepackage{lmodern}
\usepackage{textcomp}
\usepackage{lastpage}
\usepackage{geometry}
\geometry{tmargin=2cm,lmargin=2cm,includeheadfoot=True}
\usepackage{float}
\usepackage{hyperref}
\usepackage{ragged2e}
\usepackage{graphicx}
\usepackage{subcaption}
\usepackage{longtable}
\usepackage[yyyymmdd,hhmmss]{datetime}
\usepackage{enumitem}
\usepackage{placeins}
\usepackage[parfill]{parskip}
\usepackage[mainaux]{rerunfilecheck}

\begin{document}
\flushright
\begin{Large}
\textbf{IEC 62209-3 Validation Results} \\
\vspace{3pt}
\textbf{GPI Model Confirmation}
\end{Large} \\
\begin{small}
\today \\ \currenttime
\end{small}
\flushleft

\section{Executive Summary}\label{sec:exec_summary}
The SAR measurement system validation procedure described in IEC 62209-3~\cite{standard} is a three step procedure the consists of a) GPI model creation, b) model confirmation, and c) the critical data space search. This automatically generated document reports on the outcome of the \textit{model confirmation}.

\input{onelinesummary}

The results of the GPI model confirmation are shown in Table~\ref{tab:summary}.

\input{metadata}

\input{summary}

\section{Introduction}\label{sec:start}

\subsection{Purpose of Report}\label{sec:intro}

This report provides the results of the \textit{model confirmation} step for the SAR measurement system validation procedure described in IEC 62209-3~\cite{standard}. The GPI model is a model that describes the expected measurement error and uncertainty for the given SAR measurement system of interest as a function of exposure parameters.

The goal of the model confirmation step is to independently confirm, using statistical methodology, the validity of the GPI model, establishing its trustworthiness or rejecting its applicability. It assesses the ability of the GPI model to generalize beyond the exposure configurations employed in its construction.

This report has been automatically generated by an online-accessible, GUI-based validation application (\url{http://sarvalidation.site/} version 1.0.5) using measured data obtained with the SAR measurement system described in Table~\ref{tab:system}.

Background and additional information on the methodology can be found in the open-access paper~\cite{gpi-paper}. The open-source software leveraged by the online application is provided with IEC 62209-3 and can be found at
\url{https://github.com/ITISFoundation/publication-IEC62209}.


\subsection{GPI Model Confirmation}\label{sec:procedure}
SAR measurement system validation in IEC 62209-3~\cite{standard} consists of these three steps:

\begin{enumerate}[label=\alph*)]
\item Gaussian process interpolation (GPI) model creation
\item GPI model confirmation
\item Critical data space search
\end{enumerate}

The results of \textit{GPI model confirmation} (b) are reported in this document. The process involves the following steps (see Annex D, Clause D.4.4):

\begin{enumerate}[label=\arabic*)]
\item The GPI software generates a set of test configurations that is independent from the configurations used to construct the GPI model, based on the GPI model of the system under validation created in step a). That confirmation test set encompasses 50 configurations, which is sufficient to confirm the GPI model and has an associated measurement effort in the order of one day.
\item These model confirmation measurements are performed by an independent test lab. The lab is free to decide on the order in which the test configurations are measured, e.g., to optimize testing time and reduce manual interventions.
\item The test lab enters the data into the analysis software, which determines whether the model is successfully confirmed. If the confirmation fails, the lab may go back to step 1) to refine the model with the newly obtained data. If the test lab is not satisfied with the GPI model confirmation, they may contact the system manufacturer for assistance.
\end{enumerate}
Successful GPI model confirmation is the precondition for moving ahead with step c).

\subsection{Success Criteria}\label{sec:quantities}

All of the following criteria must be met for the model confirmation step to be successfully concluded:

\begin{itemize}
\item \textbf{acceptance}\\ for the system to be considered acceptable for use in accordance with IEC 62209-3, all test configuration SAR deviations ($\Delta SAR_j, j$ = 1 to $N$) between the measured SAR values and the numerical SAR target values (specified by IEC) must be within the acceptance criteria given in Clause D.4.7 of~\cite{standard}:

\[
-U < r_{s,j} < +O, ~~~~~~~~ j = 1~..~N
\]

where $r_{s,j}$ is the linear deviation between the measured SAR ($SAR_{m,j}$) and the numerical target ($SAR_{num,j}$) given by:

\[
r_{s,j} = 100~\% \times \Big(\frac{SAR_{m,j} - SAR_{num,j}}{SAR_{num,j}}\Big)
\]

and the deviation in dB is $\Delta SAR_j = 10 \times log_{10} (r_{s,j})$. The error bounds [-$U$, +$O$] are given by:

\[
+O = 2 \times u_s + 15~\%
\]

\[
-U = -100 \times \frac{2 \times u_s + 15~\%}{100 + 2 \times u_s + 15~\%}
\]

where 2 x $u_s$ is the reported measurement uncertainty of the SAR measurement with a 95~\% confidence level.

When expressed in dB, the error bounds are equal, meaning that $10 \times log_{10}(O) = 10 \times log_{10}(U)$, so the requirement simplifies to:

\[
\big|\Delta SAR_j\big| = 10 \times log_{10}(+O)
\]

\item \textbf{normality: p $\ge$ 5\,\%}\\ the result of the Shapiro-Wilk test for normality must reject the hypothesis that the SAR deviations of the measured test configurations are not normal distributed (p-value greater than 5\,\%).
\item \textbf{location $\in$ [-1, 1]}\\
measured deviations of the test data collected as part of the model confirmation should be substantially similar to those of the configurations used for model construction. A Q-Q plot is used to compare the shapes of the two distributions. The location of the Q-Q plot must be within the [-1, 1] range.
\item \textbf{scale $\in$ [0.5, 1.5]}\\
measured deviations of the test data collected as part of the model confirmation should be substantially similar to those of the configurations used for model construction. A Q-Q plot is used to compare the shapes of the two distributions. The scale of the Q-Q plot must be within the [0.5, 1.5] range.
\end{itemize}

\section{Model Confirmation}

\subsection{Limits of Relevant Exposure Parameter Space}
The test configurations for model confirmation were generated for measurement on the system detailed in Table~\ref{tab:system} according to the parameters in Table~\ref{tab:params}, which therefore define the extent of the exposure parameter space for which the GPI model can be considered to be confirmed.

\input{sample_parameters}

\FloatBarrier
\subsection{Performance on Acceptance Criteria}
The test data for the GPI model confirmation are shown in Figure~\ref{fig:confirm-acc}. The complete details on the exposure conditions and measurement results are tabulated in Table~\ref{tab:test} in Appendix~\ref{sec:test-data}.
The obtained deviations ($\Delta SAR_{10g}$) of the test configuration measurements from the target values are shown in Figure~\ref{fig:confirm-acc}, along with the acceptance thresholds. The pass/fail result is shown in Table~\ref{tab:acceptance}.

\input{acceptance}

\begin{figure}[H] \centering
\includegraphics[width=\textwidth]{images/model-confirm-acceptance.png}
\caption{Deviations of the measured $SAR_{10g}$ from the target values for the confirmation test configurations. The deviations are compared to the maximum permissible errors (mpe; dashed lines). Blue dots are inside the mpe. Any red dots are outside the mpe.} \label{fig:confirm-acc}
\end{figure}

\FloatBarrier
\subsection{Normality}

This part of the analysis evaluates how well the GPI model established in the first step generalizes to new exposure conditions. If the GPI model is generally valid, the deviations between measured and target SAR values are expected to be non-systematic and normally distributed. Deviations from a random normal distribution would therefore be indicative of an unreliable GPI model that insufficiently reproduces reality.

The Shapiro-Wilk test is applied to test for normality. Table~\ref{tab:normality} shows the p-value from the Shapiro-Wilk test and the pass/fail result.

\input{normality}

\FloatBarrier
\subsection{Similarity}

The QQ-plot in Figure~\ref{fig:confirm-qqplot} compares the experimentally found probability distribution of the deviations between the GPI model prediction and the measurement results (percentiles) against the expected standard normal distribution. The location and scale of the fitted linear relationship should be within acceptance ranges if the GPI model provides a valid representation of the performance of the measurement system under test. The success criteria, outcomes, and pass/fail classification are shown in Table~\ref{tab:qq}.

\input{similarity}

\begin{figure} \centering
\includegraphics[width=\textwidth]{images/model-confirm-qqplot.png}
\caption{QQ-plot of the SAR deviations, comparing the probability distributions of the created validation model and the test set. These SAR deviations are expected to be normally distributed. The linear regression line (red) through the resulting points (blue) should be close to the reference line (black).} \label{fig:confirm-qqplot}
\end{figure}

\FloatBarrier
\begin{thebibliography}{9}
\bibitem{standard}
IEC 62209-3, ``Measurement procedure for the assessment of specific absorption rate of human exposure to radio frequency fields from hand-held and body-mounted wireless communication devices - Part 3: Vector measurement-based systems (Frequency range of 600 MHz to 6 GHz)'', Committee Draft, February 2023.
\bibitem{gpi-paper}
C. Bujard, E. Neufeld, M. Douglas, J. Wiart, N. Kuster, ``A Gaussian-process-model-based approach for robust, independent, and implementation-agnostic validation of complex multi-variable measurement systems: application to SAR measurement systems,'' online \url{https://arxiv.org/abs/2211.12907}, uploaded April 12, 2023.
\end{thebibliography}

\FloatBarrier
\newpage
\appendix
\section{Test Data Set Configurations and Measurement Outcomes} \label{sec:test-data}

\input{sample_table}

\end{document}
